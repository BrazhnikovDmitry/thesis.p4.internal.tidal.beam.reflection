%\documentclass[12pt]{article}
%\usepackage[margin=1 in, head=0.9 in]{geometry}
%\usepackage{fancyhdr}
%\usepackage{listings}
%\usepackage{caption}
%\usepackage{color}
%\usepackage{xcolor}
%\usepackage{caption, apacite}
%\DeclareCaptionFont{white}{\color{white}}
%\DeclareCaptionFormat{listing}{\colorbox{gray}{\parbox{\textwidth}{#1#2#3}}}
%\captionsetup[lstlisting]{format=listing,labelfont=white,textfont=white}
%\usepackage{graphicx}
%\usepackage{amsmath, amssymb, amsthm}
%\usepackage[all,cmtip]{xy}
%\pagestyle{fancy}
\input{/home/dmitry/Work/Research/thesis/FINALE/settings.tex}

\begin{document}

\title{Variability of reflection and scattering of internal tidal beam from Tasman Continental slope}
\maketitle

\section{Abstract}
The Tasman internal tide beam is incident upon Tasmania continental slope. The beam is partially dissipated by means of scattering into higher modes and nonlinear processes. And partially reflects. These two phenomena depend primarily on offshore characteristics of the beam, i.e. its phase and interaction with ETP. This will defined variability of the energies in reflected wave and hence, scattered. Local mesoscale (EAC) was found to have second order effect. Additionally, it is possible that the incident internal tide

\section{Introduction}
The beam strikes the shelf. While off shore the beam is more or less homogeneous due to presence of Cascade seamount there is a focusing which leads to maxima. One is located right behind the seamount, and the other - more north. These are the sites for field program. The scattering problem consists of reflected signal, high mode generation and dissipation. We would like to address budget calculations and their variability.

\section{Numerical experiments}
It is used Regional Ocean Modeling System. The numerical domain covers southern Tasman Sea basin with grid spacing of $1/32^{\circ}$ and 50 vertical $\sigma$-levels. To simplify results $M_2$-harmonic forced as boundary condition.\\
Several cases of different background conditions is considered (Table 1). In the simple setting no background conditions are prescribed with spatially uniform stratification  representative of climatological mean. Other simulations were initialized with HYCOM hindcasted three dimensional fields of horizontal currents and stratification. On the boundaries the same fields were forced throught simulation. On the top MERRA fields were given through insolation, air temperature, evaporation-precipitation rates and wind stresses. These parameters were taken for the given in table periods.\\
The simulation were sampled after 10 days of spin up that mode-1, 2 internal tide reached stationary behavior.\\
To outline scattering energetics additionally MITgcm simulation was done with similar background conditions as 'Uni' and same resolution.\\
\begin{table}
 \caption{Carried out numerical experiments}
 \begin{tabular}{ |p{3cm}||p{5cm}|p{5cm}|  }
 \hline
 \multicolumn{3}{|c|}{Numerical experiments used in this study} \\
 \hline
Experiment Name & Dates & Comments \\
 \hline
Uni & NA & Homogeneous background condtions \\
2012 &   Janaury 1st - January 15th 2012 & NA \\
2013 &   Janaury 1st - January 15th 2013 & NA \\
2014 &   Janaury 1st - January 15th 2014 & NA \\
2015 &   Janaury 1st - January 15th 2015 & NA \\
2015\_TBEAM &   Janaury 1st - March 1st 2015 & to cover field period \\
2013\_274 &   DOY: 274 - 289 - 2013 & to test generation hypothesis \\
2015\_074 &   DOY: March 1st - March 15th 2015 & to test generation hypothesis \\
 \hline
\end{tabular}
\end{table}

The following analysis was carried on the last 5 day simulations. For the long run (2015\_TBEAM) the output was sampled in 5 days non-overlapping windows.\\

\section{Analysis}
\subsection{Reflection coefficient}
\begin{figure}
\mfig[0.5]{bird_view_ddec_1.png}
\mfig[0.5]{bird_view_ddec_2.png}
\end{figure}

\subsection{Budget}
\begin{figure}
\mfig[0.3]{budget_roms_var.pdf}
\end{figure}
%\subsection{Estimate from inverse}

\section{Reasons for variation}
\subsection{Diffraction by seamount and incident angle}
\subsection{Slope mode and local generation}
\subsection{Mesoscale}
\subsection{Resonance at NRS???}

\section{Discussion}
\section{Conclusions}

\newpage
\section*{TO DO LIST}
\begin{itemize}
\item How the slope mode excites?
\item Slope mode characteristics?
\item Offshore reflection coefficient
\item Interaction with ETP
\item Interaction with eddy
\end{itemize}

\bibliographystyle{apacite}
\bibliography{/home/dmitry/Bibtex_lib/}

\end{document}