%\documentclass[12pt]{article}
%\usepackage[margin=1 in, head=0.9 in]{geometry}
%\usepackage{fancyhdr}
%\usepackage{listings}
%\usepackage{caption}
%\usepackage{color}
%\usepackage{xcolor}
%\usepackage{caption, apacite}
%\DeclareCaptionFont{white}{\color{white}}
%\DeclareCaptionFormat{listing}{\colorbox{gray}{\parbox{\textwidth}{#1#2#3}}}
%\captionsetup[lstlisting]{format=listing,labelfont=white,textfont=white}
%\usepackage{graphicx}
%\usepackage{amsmath, amssymb, amsthm}
%\usepackage[all,cmtip]{xy}
%\pagestyle{fancy}
\input{/home/dmitry/Work/Research/thesis/FINALE/settings.tex}

\begin{document}

\title{Variability of reflection and scattering of internal tidal beam from Tasman Continental slope}
\maketitle

\begin{itemize}
\item Why did you work on this problem?
This part describes what happens with the internal tidal beams as it encounters Tasman slope. There is a great deal of complexity in reflection, scattering and energy losses. For me it was interesting to find a way to obtain a picture which will embrace these processes and will show how the coherent tidal beam disappears.

\item What did you find out?
The beam losses its energy in several spots with distinctive reflective properties: the northern is being highly reflective becoming some times super reflective. And the southern portion is more dissipative, the mode-1 scatters into higher modes more efficient. Though at latter spot the reflection is close to half, under some conditions it can become much larger. The reason for superreflectivity and higher energy lost is the same, under \textbf{...} the slope mode is excited. It couples with local barotropic forcing causing higher reflected signal at the northern site, but at the southern to trapped scattering.\\
These two phenomena depends on the beam's characteristics which are shaped by generation at Macquarie Ridge and mesoscale field. The first part was described before, but the second appears as refraction of the beam by the offshore eddies.

\item How did you tackle it?
Conversion and energy flux.

\item How do you know your results are valid?
These results partially correspond to field observations.
\item How do your results fit into the big picture?
In many ways: a) incoherence of the tidal waves on slope; b) variable diffusivities; c) close coupling to along slope bathymetry which was not described before.

\item Is further work needed?
Three dimensional theoretical theory of internal tide scattering from bathymetry
\end{itemize}

\section{Abstract}
The Tasman internal tide beam is incident upon Tasmania continental slope. The beam is partially dissipated by means of scattering into higher modes and nonlinear processes. And partially reflects. These two phenomena depend primarily on offshore characteristics of the beam, i.e. its phase and interaction with ETP. This will defined variability of the energies in reflected wave and hence, scattered. Local mesoscale (EAC) was found to have second order effect.

\section{Introduction}
The beam strikes the shelf. While off shore the beam is more or less homogeneous due to presence of Cascade seamount there is a focusing which leads to maxima. One is located right behind the seamount, and the other - more north. These are the sites for field program. The scattering problem consists of reflected signal, high mode generation and dissipation. We would like to address budget calculations and their variability.

\section{Analysis}

\newpage
\section*{TO DO LIST}
\begin{itemize}
\item How the slope mode excites?
\item Slope mode characteristics?
\item Offshore reflection coefficient
\item Interaction with ETP
\item Interaction with eddy
\end{itemize}

\bibliographystyle{apacite}
\bibliography{/home/dmitry/Bibtex_lib/}

\end{document}